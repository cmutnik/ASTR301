%[prl] it will mess up section numbers but puts our emails in the bibliography section
%[eqsecnum] Number equations by section
\documentclass[aps,prb,twocolumn,superscriptaddress]{revtex4-1}
\usepackage{graphicx}  % this is the up-to-date package for all figures
\usepackage{amssymb}   % for math
\usepackage{verbatim}  % for the comment environment
\usepackage{color}
\usepackage{subcaption} % for subcaptions on side-by-side figures
\usepackage{float}	% allows use of 'H' command

% This allows you to use '\cref{}' to reference sections with the symbol
\usepackage{cleveref}
\crefname{section}{\S}{\S\S}%{§}{§§}


%\usepackage{cite}	% to use .bib file
%\usepackage{hyperref}	% needed to add hyperlinks
%\hypersetup{
%  colorlinks=true,
%  linkcolor=blue,
%  filecolor=magenta,
%  urlcolor=cyan,
%}
\bibliographystyle{apsrev}

% For Numbered Sections---------------------------------------------------------------
% Usual (decimal) numbering
\renewcommand{\thesection}{\arabic{section}}
\renewcommand{\thesubsection}{\thesection.\arabic{subsection}}
\renewcommand{\thesubsubsection}{\thesubsection.\arabic{subsubsection}}
%% Fix references
%\makeatletter
%\renewcommand{\p@subsection}{}
%\renewcommand{\p@subsubsection}{}
%\makeatother
% ------------------------------------------------------------------------------------


% these are some custom control of the page size and margins
% \topmargin= 0.2in  % these 1st two may be needed for some computers
% \textheight=8.75in
\textwidth=6.5in
%\oddsidemargin=0cm
%\evensidemargin=0cm

% this is where the actual document itself (rather than control statements) begins:

\begin{document}



\title{Evaluating the Pan-STARRS Variability Parameter}


%\input author_list.tex

%\author{Daichi Hiramatsu\thanks{dhiramat@hawaii.edu}}
%\author{Corey Muntik\thanks{cmutnik@hawaii.edu}}

\author{Daichi Hiramatsu}
\email{dhiramat@hawaii.edu}
\author{Corey Mutnik}
\email{cmutnik@hawaii.edu}
\affiliation{Department of Physics \& Astronomy, \\
University of Hawaii at Manoa}
%\affiliation{Department of Physics \& Astronomy, \\University of Hawaii at Manoa,\\2505 Correa Rd, Honolulu, HI, 96822, USA}
%\altaffiliation{Observational Astronomy 301}



\begin{abstract}
\textbf{By Thursday (4/18) we need:} well thought out section titles and plots that show all the points we wanna make\\

remake prob(f) plot with all 300,000 stars (not only 80,000)\\

LS analysis on ATLAS Pathfinder Telescope data, verified PS variability criteria
\end{abstract}

\maketitle    




\section{Introduction}

\begin{itemize}
	\item{} why we care
	\item{} what made us care about this project
	\item{} NO structure / distance stuff (maybe put it in looking forward section at end)
	\item{} talk about PS catalog
	\item{} variability surveys (discuss other attempts to measure variables across the sky)
	\item{} why are variables interesting
	\item{} why do we want to find variables and care about where they are located
	\item{} Summary: we ran LS, analyzed stars, why did we do it all
	\item{} Mention what will be discussed: ``in section 2 we describe the observations we used...''
\end{itemize}



\section{ATLAS PathFinder 1 Observations}
\begin{itemize}
	\item{} we used data from ATLAS
	\item{} supplemented with ATLAS data [REF TONRY] (possibly make this s subsection)
%	\item{} exposure time
%	\item{} observation dates
	\item{} what was the weather like during observations
	\item{} PSF FWHM variations (only include if we discuss crowding)
	\item{} `we recieved the reduced image data from the ATLAS pipeline; which gave us RA, Dec, mag, etc...'
	\item{}\url{http://fallingstar.com/how_atlas_works.php}
\end{itemize}


~[VERIFY CORRECT CITATIONS:\\% $http://rsta.royalsocietypublishing.org/content/371/1992/20120269$]\\
%In determining the variability of stars in our FOV, data from the $gri Project$~\cite{gri} was used.\\
Initially, determination of variability was going to be achieved using data collected by the $gri Project$~\cite{gri}.  The $gri Project$ is [EXPLAIN]...\\
In order to reduce aliasing, extra observations needed to be made.  Observation procedures are discussed in~\cref{sec:data}.
~[PATHFINDER USED FOR GRI DATA...the reduction process is discussed at length Tonry in...cite]\\
We received the reduced image data from the ATLAS pipeline~\cite{PSpipe}~\cite{tonrypipe}
%\cite{PSimgpipe,*tonrypipe}


\subsection{Data}\label{sec:data}
%\begin{itemize}
%	\item{} how we got mags out of data...
%	\item{} $l^{II}=202^{\circ}$
%	\item{} $b^{II}=\pm5$
%\end{itemize}
% from old paper
\iffalse
	\begin{itemize}
	 \item{} split observations into 1 $deg^2$ chunks
	 \item{} isolated groups s.t. each one is a star with 12 or more obs
	 \item{} $--$ more than 12 detections to be a star
	 \item{} $--$ any sq deg that has more than one star
	 \item{} this reduced 1300 $deg^2$ observation data down to ~300
	 \item{} before variability params: 1531417 stars in field
	 \item{} for variability parameters
	 \item{} $--$ log(average(upper quartile)) - log(average(lower quartile))
	 \item{} $--$ expect variation to go at .2* mag (from sqrt noise)...so subtract .2mag to get the logritmic statistic
	 \item{} sorted biggest (most variable?) to smallest (least variable?)
	 \item{} ran LS on 80,000 most variable stars, rather than full star groupings (over 1million)
	\end{itemize}
\fi



Is it $b^{II}=+5^{\circ}$ or $b=+5^{\circ}$??\\
~[NEED TO DEFINE FOV]\\
~[VERIFY RA DEC RANGE (NOT WHAT WAS PUT INTO SIMBAD)]\\
~[NEED TO CITE TONRY FOR OUR OBSERVATIONS]\\
~[MENTION THAT OUR OBS NECESSARY TO REDUCE ALIASING]\\

Using the Pathfinder telescope, observations were made at two galactic latitudes ($b^{II}=\pm5^{\circ}$) and spanned a range of galactic longitudes ($202^{\circ} < l^{II} < 232^{\circ}$).  
Exposures were collected for 20~s and separated by $3^{\circ}$ longitudinally.  For implementation by the Pathfinder telescope, a conversion to RA and Dec was made; giving a range of $93 < RA < 119^{\circ}$ and $-20^{\circ} < Dec < 13^{\circ}$, as shown by Figure~\ref{fig:simoverlap}.  To account for the 0.05~$^{\circ}$ gap between the detectors, a 0.1~$^{\circ}$ offset in RA was implemented on every other night.  
%Ten observations per night were collected for 20 nights, spanning 3/8/16-3/27/16.
%Observations were collected between 3/8/16-3/27/16.  During this 20 day period, we collected 10 observations per night.  
Spanning 20 nights, 10 observations a night were collected on 3/8/16-3/27/16.  Luckily all of these nights had weather perfectly attuned for observations.  Half a night of observations were lost on 3/19/26, due to a crash of the server controlling the telescope.


%0.1 deg offset in RA, accounts for 0.05 deg gap between the two detectors.
~\\~[dRA = dangle / cos(Dec)]
~\\~[EXPLAINS EXACT OBSERVATIONAL PLAN...NECESSARY?]\\
Observations were traced out by moving the FOV by $3^{\circ}$ longitudinally, starting at $b^{II}=-5^{\circ}$, $l^{II}=202^{\circ}$ and ending at $b^{II}=-5^{\circ}$, $l^{II}=232^{\circ}$.  Once observations at $b^{II}=-5^{\circ}$ were complete, the FOV was shifted to $b^{II}=+5^{\circ}$, $l^{II}=232^{\circ}$.
%Observations were traced out by taking the first exposure at $b^{II}=-5^{\circ}$ and $l^{II}=202^{\circ}$, moving the FOV longitudinally by $3^{\circ}$
%Starting at $b^{II}=-5^{\circ}$ and $l^{II}=202^{\circ}$, exposures were collected for 20~s
%Observations were made by starting at $b^{II}=-5^{\circ}$ and $l^{II}=202^{\circ}$, moving the ...Each exposure was collected for 20
%Data was collected for \\
%Starting at $b^{II}=-5^{\circ}$ and $l^{II}=202^{\circ}$ \\
%20~s exposures were collected\\


%\begin{figure}[H]
%	\centering
%	%\caption{prob(f)}
%	\textbf{Sorting Pattern}\par\medskip
%		\includegraphics[width=3.35in]{figures/fromJT/sortedFOV.png}
%	\caption{\it \small{Stars were grouped in the pattern shown (down the collected observations)}}
%	\label{fig:sortpat}
%\end{figure}


\iffalse
\subsection{Object Cuts}
Threw away stars with magnitude error of 0.\\
After isolating `good' observations we threw away any stars with less than 12 observations: could have fallen on the edge of each 1 $deg^{2}$ FOV, fallen on portion of detector with bad PSF, etc\\
needed 12 observations to reduce / eliminate aliasing\\
 * of all 1.5 billion stars in our FOV we scanned the most variable (logPr(rnd)<=45)
  - located: {{{cd /home/rr_lyrae/logprob_below_neg45_take2}}}
  - 1558 stars
 * new grouping list that DH wanted (after removing aliased regions)
  - located: {{{cd /home/rr_lyrae/alias_rm_grps}}}
  - 776 stars
\fi



\section{Constructing Stellar Light Curves}

% check out resource:
% https://www.aavso.org/lcg

\begin{itemize}
	\item{} how we selected stars (12+ obs, 1x1 deg$^2$, etc)
\end{itemize}

The selection process began


% merged these figures into side by side (below)
%\begin{figure}[H]
% \centering
% 	\includegraphics[width=3.35in]{figures/LC/star7_97_98_0_1.png}
% \caption{\it \small{Light Curve of variable star, using all collected and ATLAS data.}}
% \label{fig:LC7}
%\end{figure}
%\begin{figure}[H]
% \centering
% 	\includegraphics[width=3.35in]{figures/LC/star7_97_98_0_1_restricted.png}
% \caption{\it \small{Light Curve of variable star, using only collected data (without incorporating extra ATLAS data).}}
% \label{fig:restrictedLC7}
%\end{figure}



\begin{figure*}
	\centering
	\begin{subfigure}{.5\textwidth}
	  \centering
	  \includegraphics[width=1\linewidth]{figures/LC/star7_97_98_0_1.png}
		\caption{\it \small{ }}
		\label{fig:LC7}
	\end{subfigure}%
	\begin{subfigure}{.5\textwidth}
	  \centering
			\includegraphics[width=1\linewidth]{figures/LC/star7_97_98_0_1_restricted.png}
		\caption{\it \small{ }}
		\label{fig:restrictedLC7}
	\end{subfigure}
	\caption{\it \small{Light Curve of a variable star.  Panel `(a)' shows a light curve constructed using all collected and ATLAS data.  Panel `(b)' is a restricted selection of `(a)', not showing any observations made by ATLAS.}}
	\label{fig:LC}
\end{figure*}



% clean up section title
\section{Fast Lomb-Scargle Periodogram}

%------------- KILL subsection and make periodgram the intro to LS
%\subsection{Periodogram Analysis}
%\section{Identifying Variable Stars Using Lomb-Scargle}
~[INTRO TO WHAT IS PERIODOGRAM...WE CONSTRUCT PERGRAMS USING LS METHOD]
\begin{itemize}
	\item{} extract variability from LS
	\item{} describe how it works and why we used LS
	\item{} major aliasing at 1 day and 0.5 day periods
	\item{} things that fall at at -50 (in Figure~\ref{fig:quartiles}) means that those are VERY probably variable stars
	\item{} roughly \_\_\_\_\_ stars fell at -50 in Figure~\ref{fig:quartiles}
	\item{} 315,992 stars tested for variability
	\item{} other stars (outside of 315,992) are statistically unlikely to be variable
\end{itemize}
%-------------

\begin{itemize}
	\item{} brief explanation of the fast LS
	\item{} period search range to avoid aliasing
	\item{} compute gri periods individually and compare each other 
	\item{} Fourier series fit to extract variable types
\end{itemize}



\begin{figure}[H]
 \centering
 	\includegraphics[width=3.35in]{figures/fromJT/probf.png}
 \caption{\it \small{prob(f) of 80,000, most variable, stars LS was run on}}
 \label{fig:quartiles}
\end{figure}

\begin{figure}[H]
 \centering
 %\caption{prob(f)}
 \textbf{prob(f)}\par\medskip
 	\includegraphics[width=3.35in]{figures/log_period_diff-50.png}
 \caption{\it \small{prob(f) of 80,000, most variable, stars LS was run on}}
 \label{fig:quartpy}
\end{figure}


\subsection{Fourier Series}

\begin{itemize}
	\item{} Fourier series fit to extract variable types
\end{itemize}

\begin{itemize}
		\item{} explanation of period error estimation using Fourier series
	\end{itemize}

	\begin{figure}[!]
	 \centering
	 %\caption{prob(f)}
	 	\includegraphics[width=3.35in]{figures/FSP1_g_LC_rrrtest_p5_grp19.png}
	 \caption{\it \small{Type ab}}
	 \label{fig:Typeab}
	\end{figure}

	\begin{figure}[!]
	 \centering
	 %\caption{prob(f)}
	 	\includegraphics[width=3.35in]{figures/FSP1_g_LC_rrrtest_p5_grp7.png}
	 \caption{\it \small{Type c}}
	 \label{fig:Typec}
	\end{figure}
	
	\begin{figure}[!]
	 \centering
	 %\caption{prob(f)}
	 	\includegraphics[width=3.35in]{figures/rPLCneg45E_limit2_grp_109+01_09693.png}
	 \caption{\it \small{Type ?}}
	 \label{fig:Typeq}
	\end{figure}

	\begin{figure}[!]
	 \centering
	 %\caption{prob(f)}
	 	\includegraphics[width=3.35in]{figures/rPLCneg45E_limit2_grp_116-11_03117.png}
	 \caption{\it \small{Blazhko Effect (period doubling)}}
	 \label{fig:Blazhko}
	\end{figure}

\iffalse

	\subsection{Period Error Estimation}

	\begin{itemize}
		\item{} explanation of period error estimation using Fourier series
	\end{itemize}

	\begin{figure}[H]
	 \centering
	 %\caption{prob(f)}
	 	\includegraphics[width=3.35in]{figures/FSP1_g_LC_rrrtest_p5_grp19.png}
	 \caption{\it \small{Type ab}}
	 \label{fig:Typeab}
	\end{figure}

	\begin{figure}[H]
	 \centering
	 %\caption{prob(f)}
	 	\includegraphics[width=3.35in]{figures/FSP1_g_LC_rrrtest_p5_grp7.png}
	 \caption{\it \small{Type c}}
	 \label{fig:Typec}
	\end{figure}
	
	\begin{figure}[H]
	 \centering
	 %\caption{prob(f)}
	 	\includegraphics[width=3.35in]{figures/rPLCneg45E_limit2_grp_109+01_09693.png}
	 \caption{\it \small{Type $\?$}}
	 \label{fig:Typeq}
	\end{figure}

	\begin{figure}[H]
	 \centering
	 %\caption{prob(f)}
	 	\includegraphics[width=3.35in]{figures/rPLCneg45E_limit2_grp_116-11_03117.png}
	 \caption{\it \small{Blazhko Effect (period doubling)}}
	 \label{fig:Blazhko}
	\end{figure}
\fi



\section{Results}
\begin{itemize}
	\item{} compare with PS~\cite{PSdata}
	\item{} density/distribution of variables in sky
	\item{} (what LS gave us for our catalog)
	\item{} put a table with ~5 stars, to show off part of catalog
\end{itemize}

% From old paper
\subsection{Simbad Completeness}
	\begin{itemize}
		\item{} Pull established RR list from Simbad
		\item{} Pull other variable data from simbad, too
		\item{} Compare list of observed RR to catalogs
		\item{} Is anyone actually reading this outline, this bullet point serves no purpose
		\item{} Wow, its sad how little Jeff did since class began (especially after JT gave him the code to do it a month ago) - 6 obs x 4 nights = January-April work period haha
		\item{} Establish completeness with Simbad
	\end{itemize}
~[FIX SIMBAD CITATION]\\
In order to evaluate the completeness of our results, comparisons needed to be made to other variable star catalogs.  Simbad~\cite{simbad} provided a list of variable stars within our FOV.  Pulsating sources encompasses all variable objects.  \\
~[OF THESE WE MATCHED...RR LYRAE...REF FIGURE]\\


~[review what he did, but heres a summary:]\\
~[48 simbad objects overlay our FOV]\\
~[33 have an entry in lsum.dat]\\
~[Of 15 that overlap w/o lsum entry]\\
~[14 are defective]\\
~[at least one filter has less than 10 detections]\\
~[=> it's a survey problem, not an LC problem]\\
~[P(rand) for the 33 that do occur in lsum]\\
~[27/33 show up as high probability variables]\\



\begin{figure}[H]
 \centering
 %\caption{prob(f)}
 %\textbf{Simbad Completeness}\par\medskip
 	\includegraphics[width=3.35in]{figures/simbadoverlap.png}
 \caption{\it \small{Observation path with Simbad pulsators in blue and RR Lyrae in green.}}
 \label{fig:simoverlap}
\end{figure}



%\begin{figure}[H]
%	\centering
%		\includegraphics[width=3.35in]{figures/aitoff/Obs_PS_lsum_aitoff_map.png}
%		\caption{\it \small{Aitoff projection of observed and PS RR Lyrae candidates.  Blue are candidates from PS that >= 0.05, green are PS candidates that >= 0.2., observed data in red.}}%
%	\label{fig:aitoff_nosimbad}
%\end{figure}


\section{Discussion}
\begin{itemize}
	\item{} Evaluation of PS criteria
	\item{} what went wrong
	\item{} what could have gone better
	\item{} future outlook
	\item{}~~~we could map spiral arms using x y and z
\end{itemize}

	

%\begin{figure}[h!]
%	\centering
%	\begin{subfigure}{.5\textwidth}
%	  \centering
%	  \includegraphics[width=1\linewidth]{figures/aitoff/simbad_alpha_is_1.png}
%	  \caption{\it \small{Aitoff map.}}
%	  \label{fig:aitoff_map_simbad}
%	\end{subfigure}%
%	\begin{subfigure}{.5\textwidth}
%	  \centering
%	  \includegraphics[width=1\linewidth]{figures/aitoff/Obs_PS_sim_lsum_aitoff_map.png}
%	 \caption{\it \small{Aitoff projection}}%
%	 \label{fig:aitoff_projection_simbad}
%	\end{subfigure}
%	\caption{\it \small{Aitoff projection of observed and PS RR Lyrae candidates.  Blue are candidates from PS that >= 0.05, yellow are PS candidates that >= 0.2., observed data in red, simbad in black.}}
%	\label{fig:aitoff_simbad}
%\end{figure}	


\section{Summary and Conclusions}
~[TALK ABOUT IT]


\section{Resources from website (remove this section)}
\begin{itemize}
	\item{} $ay301.ifa.hawaii.edu$
	\item{} $/ay301/atlas = location for all the gri Project data and programs$
	\item{} $/ay301/script = transcripts from class computer activities$
	\item{} $/ay301/tmp = generic temporary area$
	\item{} $gri Project$
	\item{} $/home/gri_project = location for manuscript and figures$
	\item{} $Las Cumbres LCOGT (Faulkes 2m telescope on Haleakala)$
	\item{} $ATLAS (fallingstar.com)$
	\item{} $0.5m telescope with c,o,B,V,R,I,H-alpha filters$
	\item{} $0.18m Pathfinder telescope with g,r,i filters$
	\item{} $0.06m 35mm piggy-back camera with RGB$
	\item{} $0.01m 35mm fisheye camera$
\end{itemize}



\section*{Acknowledgments}
We would like to thank \input acknowledgement.tex  % input acknowledgement





\setlength{\parindent}{0cm}

\bibliography{biblio}


%\begin{thebibliography}{99}  % the trailing 99 controls some obscure format--just use	
%~\\% needed for spacing	
%\bibitem{PSimgpipe} Magnier E 2006 The Pan-STARRS PS1 image processing pipeline. In \textit{Proceedings of the Advanced Maui Optical and Space Surveillance Technologies Conf.} (ed. Ryan S), \textit{Wailea, Maui, HI, 10–14 September 2006,} p. E5. Kihei, HI: The Maui Economic Development Board.\\	
%\bibitem{simbad} Simbad Database	
%\bibitem{tonrypipe} Tonry JL, et al. 2012 The Pan-STARRS1 photometric system. \textit{Astrophys. J.} \textbf{750}, 99. doi:10.1088/0004-637X/750/2/99\\	
%\bibitem{gri} Tonry JL, er al. 2016 (personal communications January-April 2016)\\	
%\end{thebibliography}


\end{document}

