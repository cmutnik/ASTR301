\documentclass[aps,prl,twocolumn,superscriptaddress]{revtex4-1}
\usepackage{graphicx}  % this is the up-to-date package for all figures
\usepackage{amssymb}   % for math
\usepackage{verbatim}  % for the comment environment
\usepackage{color}
\usepackage{subcaption} % for subcaptions on side-by-side figures
\usepackage{float}	% allows use of 'H' command
%\usepackage{hyperref}	% needed to add hyperlinks
%\hypersetup{
%  colorlinks=true,
%  linkcolor=blue,
%  filecolor=magenta,
%  urlcolor=cyan,
%}
\bibliographystyle{apsrev}


% these are some custom control of the page size and margins
% \topmargin= 0.2in  % these 1st two may be needed for some computers
% \textheight=8.75in
\textwidth=6.5in
%\oddsidemargin=0cm
%\evensidemargin=0cm

% this is where the actual document itself (rather than control statements) begins:

\begin{document}



\title{Evaluating the Pan-STARRS Variability Parameter}


\author{Daichi Hiramatsu}
\author{Corey Mutnik}
\email{dhiramat@hawaii.edu}
\email{cmutnik@hawaii.edu}
\affiliation{Department of Physics \& Astronomy, \\
University of Hawaii at Manoa,\\
2505 Correa Rd, Honolulu, HI, 96822, USA}
\altaffiliation{Observational Astronomy 301}



\begin{abstract}
\textbf{By Thursday (4/18) we need:} well thought out section titles and plots that show all the points we wanna make\\

remake prob(f) plot with all 300,000 stars (not only 80,000)\\

LS analysis on ATLAS Pathfinder Telescope data, verified PS variability criteria
\end{abstract}

\maketitle    




\section{Introduction}

\begin{itemize}
	\item{} why we care
	\item{} what made us care about this project
	\item{} NO structure / distance stuff (maybe put it in looking forward section at end)
	\item{} talk about PS catalog
	\item{} variability surveys (discuss other attempts to measure variables across the sky)
	\item{} why are variables interesting
	\item{} why do we want to find variables and care about where they are located
	\item{} Summary: we ran LS, analyzed stars, why did we do it all
	\item{} Mention what will be discussed: ``in section 2 we describe the observations we used...''
\end{itemize}

\section{ATLAS PathFinder 1 Observations}
\begin{itemize}
	\item{} we used data from ATLAS
	\item{} supplemented with ATLAS data [REF TONRY] (possibly make this s subsection)
	\item{} exposure time
	\item{} observation dates
	\item{} what was the weather like during observations
	\item{} PSF FWHM variations (only include if we discuss crowding)
	\item{} `we recieved the reduced image data from the ATLAS pipeline; which gave us RA, Dec, mag, etc...'
\end{itemize}

\subsection{Data}

\begin{itemize}
	\item{} how we got mags out of data...
	\item{} $l^{II}=202^{\circ}$
	\item{} $b^{II}=\pm5$
\end{itemize}

\begin{figure}[H]
	\centering
	%\caption{prob(f)}
	\textbf{Sorting Pattern}\par\medskip
		\includegraphics[width=3.35in]{figures/fromJT/sortedFOV.png}
	\caption{\it \small{Stars were grouped in the pattern shown (down the collected observations)}}
	\label{fig:sortpat}
\end{figure}

% from old paper
\iffalse
	\begin{itemize}
	 \item{} split observations into 1 $deg^2$ chunks
	 \item{} isolated groups s.t. each one is a star with 12 or more obs
	 \item{} $--$ more than 12 detections to be a star
	 \item{} $--$ any sq deg that has more than one star
	 \item{} this reduced 1300 $deg^2$ observation data down to ~300
	 \item{} before variability params: 1531417 stars in field
	 \item{} for variability parameters
	 \item{} $--$ log(average(upper quartile)) - log(average(lower quartile))
	 \item{} $--$ expect variation to go at .2* mag (from sqrt noise)...so subtract .2mag to get the logritmic statistic
	 \item{} sorted biggest (most variable?) to smallest (least variable?)
	 \item{} ran LS on 80,000 most variable stars, rather than full star groupings (over 1million)
	\end{itemize}
\fi



\section{Constructing Stellar Light Curves}
%\section{Constructing LightCurves for Stars}

% check out resource:
% https://www.aavso.org/lcg

\begin{itemize}
	\item{} how we selected stars (12+ obs, 1x1 deg$^2$, etc)
\end{itemize}



\begin{figure}[H]
 \centering
 	\includegraphics[width=3.35in]{figures/LC/star7_97_98_0_1.png}
 \caption{\it \small{Light Curve of variable star, using all collected and ATLAS data.}}
 \label{fig:LC7}
\end{figure}

\begin{figure}[H]
 \centering
 	\includegraphics[width=3.35in]{figures/LC/star7_97_98_0_1_restricted.png}
 \caption{\it \small{Light Curve of variable star, using only collected data (without incorporating extra ATLAS data).}}
 \label{fig:restrictedLC7}
\end{figure}




\section{Periodogram Analysis}
%\section{Identifying Variable Stars Using Lomb-Scargle}
\begin{itemize}
	\item{} extract variability from LS
	\item{} describe how it works and why we used LS
	\item{} major aliasing at 1 day and 0.5 day periods
	\item{} things that fall at at -50 (in Figure~\ref{fig:quartiles}) means that those are VERY probably variable stars
	\item{} roughly \_\_\_\_\_ stars fell at -50 in Figure~\ref{fig:quartiles}
	\item{} 315,992 stars tested for variability
	\item{} other stars (outside of 315,992) are statistically unlikely to be variable
\end{itemize}

\subsection{Fast Lomb-Scargle Periodogram}

\begin{itemize}
	\item{} brief explanation of the fast LS
	\item{} period search range to avoid aliasing
	\item{} compute gri periods individually and compare each other 
	\item{} Fourier series fit to extract variable types
\end{itemize}



\begin{figure}[H]
 \centering
 %\caption{prob(f)}
 \textbf{prob(f)}\par\medskip
 	\includegraphics[width=3.35in]{figures/fromJT/probf.png}
 \caption{\it \small{prob(f) of 80,000, most variable, stars LS was run on}}
 \label{fig:quartiles}
\end{figure}

\subsection{Fourier Series}

\begin{itemize}
	\item{} Fourier series fit to extract variable types
\end{itemize}

\subsection{Period Error Estimation}

\begin{itemize}
	\item{} explanation of period error estimation using Fourier series
\end{itemize}

\begin{figure}[H]
 \centering
 %\caption{prob(f)}
 	\includegraphics[width=3.35in]{figures/FSP1_g_LC_rrrtest_p5_grp19.png}
 \caption{\it \small{Type ab}}
 \label{fig:quartiles}
\end{figure}

\begin{figure}[H]
 \centering
 %\caption{prob(f)}
 	\includegraphics[width=3.35in]{figures/FSP1_g_LC_rrrtest_p5_grp7.png}
 \caption{\it \small{Type c}}
 \label{fig:quartiles}
\end{figure}


\section{Results}
\begin{itemize}
	\item{} compare with PS
	\item{} density/distribution of variables in sky
	\item{} (what LS gave us for our catalog)
	\item{} put a table with ~5 stars, to show off part of catalog
\end{itemize}

% From old paper
\subsection{Simbad Completeness}
	\begin{itemize}
		\item{} Pull established RR list from Simbad
		\item{} Pull other variable data from simbad, too
		\item{} Compare list of observed RR to catalogs
		\item{} Is anyone actually reading this outline, this bullet point serves no purpose
		\item{} Wow, its sad how little Jeff did since class began (especially after JT gave him the code to do it a month ago) - 6 obs x 4 nights = January-April work period haha
		\item{} Establish completeness with Simbad
	\end{itemize}
\begin{figure}[H]
 \centering
 %\caption{prob(f)}
 %\textbf{Simbad Completeness}\par\medskip
 	\includegraphics[width=3.35in]{figures/simbadoverlap.png}
 \caption{\it \small{Observation path with Simbad pulsators in blue and RR Lyrae in green.}}
 \label{fig:simoverlap}
\end{figure}



\begin{figure}[H]
	\centering
		\includegraphics[width=3.35in]{figures/aitoff/Obs_PS_lsum_aitoff_map.png}
		\caption{\it \small{Aitoff projection of observed and PS RR Lyrae candidates.  Blue are candidates from PS that >= 0.05, green are PS candidates that >= 0.2., observed data in red.}}%
	\label{fig:aitoff_nosimbad}
\end{figure}


\section{Discussion}
\begin{itemize}
	\item{} Evaluation of PS criteria
	\item{} what went wrong
	\item{} what could have gone better
	\item{} future outlook
	\item{}~~~we could map spiral arms using x y and z
\end{itemize}

	

\begin{figure}[h!]
	\centering
	\begin{subfigure}{.5\textwidth}
	  \centering
	  \includegraphics[width=1\linewidth]{figures/aitoff/simbad_alpha_is_1.png}
	  \caption{\it \small{Aitoff map.}}
	  \label{fig:aitoff_map_simbad}
	\end{subfigure}%
	\begin{subfigure}{.5\textwidth}
	  \centering
	  \includegraphics[width=1\linewidth]{figures/aitoff/Obs_PS_sim_lsum_aitoff_map.png}
	 \caption{\it \small{Aitoff projection}}%
	 \label{fig:aitoff_projection_simbad}
	\end{subfigure}
	\caption{\it \small{Aitoff projection of observed and PS RR Lyrae candidates.  Blue are candidates from PS that >= 0.05, yellow are PS candidates that >= 0.2., observed data in red, simbad in black.}}
	\label{fig:aitoff_simbad}
\end{figure}	


\section{Summary and Conclusions}
~[TALK ABOUT IT]


\section{Resources from website (remove this section)}
\begin{itemize}
	\item{} $ay301.ifa.hawaii.edu$
	\item{} $/ay301/atlas = location for all the gri Project data and programs$
	\item{} $/ay301/script = transcripts from class computer activities$
	\item{} $/ay301/tmp = generic temporary area$
	\item{} $gri Project$
	\item{} $/home/gri_project = location for manuscript and figures$
	\item{} $Las Cumbres LCOGT (Faulkes 2m telescope on Haleakala)$
	\item{} $ATLAS (fallingstar.com)$
	\item{} $0.5m telescope with c,o,B,V,R,I,H-alpha filters$
	\item{} $0.18m Pathfinder telescope with g,r,i filters$
	\item{} $0.06m 35mm piggy-back camera with RGB$
	\item{} $0.01m 35mm fisheye camera$
\end{itemize}



\setlength{\parindent}{0cm}

\begin{thebibliography}{99}  % the trailing 99 controls some obscure format--just use

%\bibitem{Sch_eq} Weisstein, Eric W. ``Schr\"{o}dinger Equation."

\end{thebibliography}


\end{document}

