\documentclass[aps,prl,twocolumn,superscriptaddress]{revtex4-1}
\usepackage{graphicx}  % this is the up-to-date package for all figures
\usepackage{amssymb}   % for math
\usepackage{verbatim}  % for the comment environment
\usepackage{color}
\usepackage{float}	% allows use of 'H' command
%\usepackage{hyperref}	% needed to add hyperlinks
%\hypersetup{
%  colorlinks=true,
%  linkcolor=blue,
%  filecolor=magenta,
%  urlcolor=cyan,
%}
\bibliographystyle{apsrev}


% these are some custom control of the page size and margins
% \topmargin= 0.2in  % these 1st two may be needed for some computers
% \textheight=8.75in
\textwidth=6.5in
%\oddsidemargin=0cm
%\evensidemargin=0cm

% this is where the actual document itself (rather than control statements) begins:

\begin{document}



\title{Transient Mapping}


\author{Daichi Hiramatsu}
\author{Corey Mutnik}
\email{dhiramat@hawaii.edu}
\email{cmutnik@hawaii.edu}
\affiliation{Department of Physics \& Astronomy, \\
University of Hawaii at Manoa,\\
2505 Correa Rd, Honolulu, HI, 96822, USA}
\altaffiliation{Observational Astronomy 301}



	      % \section is used to start a new one with a heading
\begin{abstract}

\end{abstract}

\maketitle    


\section{}


\section{3D map galaxy - var. stars}

\begin{itemize}
	\item{} Use gri data to identify variable stars
	\item{} Use Period-Luminosity relationship to get distance
	\item{} Map 3D spatial distribution
	\item{} Determine deviation of variable stars from model
	\item{} Variations arise from non-gravitational effects
	\item{} Figure out dark matter distribution
\end{itemize}

\section{Pan-STARRS Comparison}
\begin{itemize}
	\item{} download Pan-STARRS data (running)
	\item{} compare generated variable star list to PS RA and Dec
	\item{} validate observed variable stars
\end{itemize}



\section{Confirm accelerated expansion - Super Novea}
\begin{itemize}
	\item{} use PanStarrs data to identify supernova locations
\end{itemize}



%\begin{figure}[h!]
% \centering
% \includegraphics[width=3.35in]{harmOsc.png}
% % harmOsc.png: 0x0 pixel, 300dpi, 0.00x0.00 cm, bb=
% \caption{\it \small{The wavefunctions associated with the first six bound eigenstates, $n$=0 to 5, as generated by Numerov's algorithm.}}%
% \label{fig:harmOsc}
%\end{figure}

%\begin{table}[h!]
%\label{data}
%\begin{center}
%\begin{tabular}{|c|c|c|c|} 
%\hline
%n  & Predicted $E_n$  & Simulated $E_n$    & Fractional Error \\
%\hline\hline
%0  & $\frac{1}{2}\hbar\omega$ & $0.500000035\hbar\omega$ & $7.0\times10^{-8}$\\
%\hline
%\end{tabular}
%\caption{\it\small{Comparison of simulated and predicted eigenvalues}}
%\end{center}
%\end{table}




\setlength{\parindent}{0cm}

\begin{thebibliography}{99}  % the trailing 99 controls some obscure format--just use

%\bibitem{Sch_eq} Weisstein, Eric W. ``Schr\"{o}dinger Equation."

\end{thebibliography}


\end{document}

