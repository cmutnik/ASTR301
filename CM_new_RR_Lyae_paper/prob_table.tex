%[prl] it will mess up section numbers but puts our emails in the bibliography section
%[eqsecnum] Number equations by section
\documentclass[aps,prb,twocolumn,superscriptaddress]{revtex4-1}
\usepackage{graphicx}  % this is the up-to-date package for all figures
\usepackage{amssymb}   % for math
\usepackage{verbatim}  % for the comment environment
\usepackage{color}
\usepackage{subcaption} % for subcaptions on side-by-side figures
\usepackage{float}	% allows use of 'H' command

% This allows you to use '\cref{}' to reference sections with the symbol
\usepackage{cleveref}
\crefname{section}{\S}{\S\S}%{§}{§§}


%\usepackage{cite}	% to use .bib file
%\usepackage{hyperref}	% needed to add hyperlinks
%\hypersetup{
%  colorlinks=true,
%  linkcolor=blue,
%  filecolor=magenta,
%  urlcolor=cyan,
%}
\bibliographystyle{apsrev}

% For Numbered Sections---------------------------------------------------------------
% Usual (decimal) numbering
\renewcommand{\thesection}{\arabic{section}}
\renewcommand{\thesubsection}{\thesection.\arabic{subsection}}
\renewcommand{\thesubsubsection}{\thesubsection.\arabic{subsubsection}}
%% Fix references
%\makeatletter
%\renewcommand{\p@subsection}{}
%\renewcommand{\p@subsubsection}{}
%\makeatother
% ------------------------------------------------------------------------------------


% these are some custom control of the page size and margins
% \topmargin= 0.2in  % these 1st two may be needed for some computers
% \textheight=8.75in
\textwidth=6.5in
%\oddsidemargin=0cm
%\evensidemargin=0cm

% this is where the actual document itself (rather than control statements) begins:

\begin{document}



\title{Evaluating the Pan-STARRS Variability Parameter}


%\input author_list.tex

%\author{Daichi Hiramatsu\thanks{dhiramat@hawaii.edu}}
%\author{Corey Muntik\thanks{cmutnik@hawaii.edu}}

\author{Daichi Hiramatsu}
\email{dhiramat@hawaii.edu}
\author{Corey Mutnik}
\email{cmutnik@hawaii.edu}
\affiliation{Department of Physics \& Astronomy, \\
University of Hawaii at Manoa}
%\affiliation{Department of Physics \& Astronomy, \\University of Hawaii at Manoa,\\2505 Correa Rd, Honolulu, HI, 96822, USA}
%\altaffiliation{Observational Astronomy 301}



\begin{abstract}
\textbf{By Thursday (4/18) we need:} well thought out section titles and plots that show all the points we wanna make\\

remake prob(f) plot with all 300,000 stars (not only 80,000)\\

LS analysis on ATLAS Pathfinder Telescope data, verified PS variability criteria
\end{abstract}

\maketitle    



\section{Number of stars in each bin, bins denote $\rho_{RRLyrae}$}

\begin{table*}
%\begin{table}[H]
	\begin{center}
		%\begin{tabular}{ | c | c | c | c | c | c | c | c | c | c | }\hline
		\begin{tabular}{| c | c | c | c | c | c | c | c | c | c | c | c |}\hline
			SourceFile & $0.0-0.05$ & $0.05-0.1$ & $0.1-0.2$ & $0.2-0.3$ & $0.3-0.4$ & $0.4-0.5$ & $0.5-0.6$ & $0.6-0.7$ & $0.7-0.8$ & $0.8-0.9$ & $0.9-1.0$ \\ \hline
			nonvar15 & 89 & 13 & 18 & 20 & 14 & 13 & 18 & 17 & 11 & 3 & 6 \\ \hline
			var15 & 41 & 11 & 16 & 12 & 21 & 19 & 21 & 25 & 27 & 21 & 7 \\ \hline
			nonvar20 & 104 & 18 & 25 & 23 & 24 & 21 & 25 & 23 & 19 & 8 & 6 \\ \hline
			var20 & 26 & 6 & 9 & 9 & 11 & 11 & 14 & 19 & 19 & 16 & 7 \\ \hline
		\end{tabular}
	\end{center}
\caption{ \small{ProbRR for each bin\label{tab:probbRRbin}}}
\end{table*}
%\end{table}





\begin{table}[H]
	\begin{center}
		%\begin{tabular}{ | c | c | c | c | c | c | c | c | c | c | }\hline
		\begin{tabular}{| c | c | c | c | c | c | c | c | c | c | c | c |}\hline
			$0.0-0.05$ 
			$0.05-0.1$ 
			$0.1-0.2$ 
			$0.2-0.3$ 
			$0.3-0.4$ 
			$0.4-0.5$ 
			$0.5-0.6$ 
			$0.6-0.7$ 
			$0.7-0.8$ 
			$0.8-0.9$
		\end{tabular}
	\end{center}
\caption{ \small{ProbRR for each bin\label{tab:probbRRbin}}}
\end{table}

\begin{table}[H]
	\begin{center}
		\begin{tabular}{ | c | c | c | c | c | c | c | }\hline
		$\rho_{RR}$ & \# HPS & \# RR & \# Other Variable Class & \# not var & Prob(var) & Prob(RR) \\ \hline
		0.0-0.05 &&&&&& \\ \hline
		0.05-0.1 &&&&&& \\ \hline
		0.1-0.2 &&&&&& \\ \hline
		0.2-0.3 &&&&&& \\ \hline
		0.3-0.4 &&&&&& \\ \hline
		0.4-0.5 &&&&&& \\ \hline
		0.5-0.6 &&&&&& \\ \hline
		0.6-0.7 &&&&&& \\ \hline
		0.7-0.8 &&&&&& \\ \hline
		0.8-0.9 &&&&&& \\ \hline
		0.9-1.0 &&&&&& \\ \hline
		\end{tabular}
	\end{center}
\caption{ \small{RR Table 0.2 \label{tab:rrtab}}}
\end{table}



\end{document}

